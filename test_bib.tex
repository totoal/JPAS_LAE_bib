%Send a message% mnras_template.tex
%
% LaTeX template for creating an MNRAS paper
%
% v3.0 released 14 May 2015
% (version numbers match those of mnras.cls)
%
% Copyright (C) Royal Astronomical Society 2015
% Authors:
% Keith T. Smith (Royal Astronomical Society)

% Change log
%
% v3.0 May 2015
%    Renamed to match the new package name
%    Version number matches mnras.cls
%    A few minor tweaks to wording
% v1.0 September 2013
%    Beta testing only - never publicly released
%    First version: a simple (ish) template for creating an MNRAS paper

%%%%%%%%%%%%%%%%%%%%%%%%%%%%%%%%%%%%%%%%%%%%%%%%%%
% Basic setup. Most papers should leave these options alone.
\documentclass[a4paper,fleqn,usenatbib]{mnras}

% MNRAS is set in Times font. If you don't have this installed (most LaTeX
% installations will be fine) or prefer the old Computer Modern fonts, comment
% out the following line
\usepackage{newtxtext,newtxmath}
\let\Bbbk\relax 
% Depending on your LaTeX fonts installation, you might get better results with one of these:
%\usepackage{mathptmx}
%\usepackage{txfonts}

% Use vector fonts, so it zooms properly in on-screen viewing software
% Don't change these lines unless you know what you are doing
\usepackage[T1]{fontenc}
\usepackage{ae,aecompl}


%%%%% AUTHORS - PLACE YOUR OWN PACKAGES HERE %%%%%

% Only include extra packages if you really need them. Common packages are:
\usepackage{graphicx}	% Including figure files
\usepackage{amsmath}	% Advanced maths commands
\usepackage{amssymb}	% Extra maths symbols
\usepackage{multirow}

%\usepackage{subcaption}
%\usepackage{hyperref}
\usepackage[utf8]{inputenc}
\usepackage[export]{adjustbox}
\usepackage{wrapfig}
\usepackage{dutchcal}
\usepackage{braket}
\usepackage{grffile} % allow for weird plot filenames
\usepackage{xspace} % smart spaces
%\usepackage{txfonts}

\usepackage{pdflscape}
\usepackage{booktabs}

%\usepackage{rotating}

\usepackage{ulem}
%%%%%%%%%%%%%%%%%%%%%%%%%%%%%%%%%%%%%%%%%%%%%%%%%%

%%%%% AUTHORS - PLACE YOUR OWN COMMANDS HERE %%%%%

% Please keep new commands to a minimum, and use \newcommand not \def to avoid
% overwriting existing commands. Example:
%\newcommand{\pcm}{\,cm$^{-2}$}	% per cm-squared
\graphicspath{ {images/} }




%%%%%%%%%%%%%%%%%%%%%%%%%%%%%%%%%%%%%%%%%%%%%%%%%%

%%%%%%%%%%%%%%%%%%% TITLE PAGE %%%%%%%%%%%%%%%%%%%

% Title of the paper, and the short title which is used in the headers.
% Keep the title short and informative.
\title[J-LAEs bib]{J-PAS LAEs Bibliography}
% Alternatives for title:
%[RT impact on LAEs samples I : Galaxy Outflows]{Radiative Transfer impact on Lyman $\alpha$ emitters %samples I : Galaxy outflows.}
% [Lyman$\alpha$ emitters in simulations I]{Ly-alpha radiative transfer in star-forming galaxies I: a recipe for cosmological simulations.}


% The list of authors, and the short list which is used in the headers.
% If you need two or more lines of authors, add an extra line using \newauthor
%\author[K. T. Smith et al.]{
%Keith T. Smith,$^{1}$\thanks{E-mail: mn@ras.org.uk (KTS)}
%A. N. Other,$^{2}$
%Third Author$^{2,3}$
%and Fourth Author$^{3}$
%\\
% List of institutions
%$^{1}$Royal Astronomical Society, Burlington House, Piccadilly, London W1J 0BQ, UK\\
%$^{2}$Department, Institution, Street Address, City Postal Code, Country\\
%$^{3}$Another Department, Different Institution, Street Address, City Postal Code, Country
%}

\author[Alberto T]{
Alberto T\thanks{E-mail: alberto.torralba@gmail.com}
}

% These dates will be filled out by the publisher
\date{Accepted XXX. Received YYY; in original form ZZZ}

% Enter the current year, for the copyright statements etc.
\pubyear{2020}

% Don't change these lines
%\hypersetup{draft}
\begin{document}
\label{firstpage}
\pagerange{\pageref{firstpage}--\pageref{lastpage}}
\maketitle

% Abstract of the paper
\begin{abstract}


Archive of bibliography for J-PAS LAEs project.
\end{abstract}

% Select between one and six entries from the list of approved keywords.
% Don't make up new ones.
\begin{keywords}
bibliography, papers
\end{keywords}

%%%%%%%%%%%%%%%%%%%%%%%%%%%%%%%%%%%%%%%%%%%%%%%%%%

%%%%%%%%%%%%%%%%% BODY OF PAPER %%%%%%%%%%%%%%%%%%



\section*{Articles}

\begin{itemize}
    \item Historical LAEs paper. \cite{Partridge1967}
    \item J-PAS red book. \cite{benitez2014jpas}
    \item LAEs survey in SUBARU. \cite{ouchi2018}
    \item CIV and CIII lines survey. \cite{stroe2017}
    \item miniJPAS main paper. \cite{bonoli2020}
    \item Ly$\alpha$ review. \cite{ouchi2020}
    \item Star-Galaxy classification using Machine Learning in J-PAS. \cite{baqui2021}
    \item Machine learning detection of LAEs in SILVERRUSH X. \cite{ono2021}
    \item Ly$\alpha$ radiative transfer review. \cite{Dijkstra2019}
    \item LAEs \& LBGs properties. \cite{ArrabalHaro2020}
    \item Star forming models in LAEs. EW distribution. \cite{Charlot1993}
    \item OII emitters Luminosity Function. \cite{Ciardullo2013}
    \item OII LF. \cite{Gilbank2010}
    \item LF interloper correction HETDEX. \cite{Farrow2021}
    \item 2D Luminosity Function for H$\alpha$ in GAMA survey. \cite{Gunawardhana2002}
    \item miniJPAS photo-z. \cite{Hernan-Caballero2021}
    \item QSOs in ALHAMBRA. \cite{Matute2018}
    \item LAEs in J-PLUS. \cite{Spinoso2020}
    \item HETDEX LAEs LF. \cite{Zhang2021}
    \item LAEs UV continuum vs. Ly$\alpha$ luminosity relation. \cite{Santos2021}
    \item LAEs LF z=2. \cite{Konno2016}
    \item Luminosity function of LAEs z=2-6. \cite{Sobral2018}
    \item Galatic models for SF mocks. \cite{bruzual2003}
    \item Details on line fitting for eBOSS QSOs. \cite{bolton2012}
    \item J-PAS. \cite{Benitez2014}
    \item Lyman-break absorption model. \cite{Faucher-Giguere2008}
    \item SDSS DR12. \cite{Alam2015}
    \item Planck18 paper. \cite{Planck18}
    \item Are young galaxies visible? \cite{Partridge1967}
    \item LAEs at z=3.1. \cite{Gronwall2007}
\end{itemize}



%%%%%%%%%%%%%%%%%%%%%%%%%%%%%%%%%%%%%%%%%%%%%%%%%%

%%%%%%%%%%%%%%%%%%%% REFERENCES %%%%%%%%%%%%%%%%%%

% The best way to enter references is to use BibTeX:

\bibliographystyle{mnras}
%\bibliography{ref,references_linked} % if your bibtex file is called example.bib
\bibliography{my_bibliography}

% Alternatively you could enter them by hand, like this:
% This method is tedious and prone to error if you have lots of references
%\begin{thebibliography}{99}
%\bibitem[\protect\citeauthoryear{Author}{2012}]{Author2012}
%Author A.~N., 2013, Journal of Improbable Astronomy, 1, 1
%\bibitem[\protect\citeauthoryear{Others}{2013}]{Others2013}
%Others S., 2012, Journal of Interesting Stuff, 17, 198
%\end{thebibliography}

%%%%%%%%%%%%%%%%%%%%%%%%%%%%%%%%%%%%%%%%%%%%%%%%%%

%%%%%%%%%%%%%%%%% APPENDICES %%%%%%%%%%%%%%%%%%%%%





% Don't change these lines
\bsp	% typesetting comment
\label{lastpage}
\end{document}
